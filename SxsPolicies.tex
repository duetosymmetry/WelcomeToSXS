\documentclass[12pt]{article}
\usepackage[headheight=14.5pt,margin=2.5cm]{geometry}
\usepackage{amsmath,amsfonts}
\usepackage{csquotes}
\usepackage{hyperref,fancyhdr}
\usepackage{draftwatermark}
\usepackage[dvipsnames]{xcolor}
\newcommand{\comment}[1]{{\color{red}#1}}
\newcommand{\discuss}[1]{{\color{blue}#1}}
\newcommand{\action}[1]{{\color{violet}#1}}

\author{Simulating eXtreme Spacetimes Collaboration}
\date{\today \\Draft of Version 1.0}
\title{Policies}

% Make fancy headers
\pagestyle{fancyplain}
\lhead[SXS Collaboration]{SXS Collaboration}
\chead[Publication Policies]{Policies}
\rhead[\thepage]{\thepage}
\cfoot{}
% Start the document
\begin{document}
\maketitle
\section{Introduction}

The Simulating eXtreme Spacetimes (SXS) collaboration is a
multi-institutional collaboration dedicated to the simulation of
extreme spacetimes in order to predict multi-messenger signals
(gravitational wave, electromagnetic, and neutrino) from high-energy
astrophysical events.  This document describes various policies that
members of the SXS collaboration are expected to adhere to in order to
carry out effective collaborative research.  One of the goals of these
policies is to create an inclusive, supportive and harassment-free
environment that nurtures junior researchers.

\section{SXS membership}\label{sec:sxs_membership}

The SXS collaboration consists of a number of faculty and senior
researchers that have agreed to use common
\emph{resources}~(\S\ref{sec:resources}) in order to pursue
collaborative research.  The current faculty and senior researchers,
hereafter referred to as \emph{senior members}, are listed
in~\S\ref{sec:member_faculty}.

SXS \emph{members} are also the post-docs, graduate students, and
undergradutae students who are working with a SXS senior member on SXS
\emph{projects}~(\S\ref{sec:projects}).  All SXS members are expected
to adhere to the collaboration policies.  If someone is unsure of
somebody's membership status they should consult the SXS
\emph{executive committee}~(\S\ref{sec:executive_committee}).
Membership in the SXS collaboration does not guarantee authorship on
SXS publications.  SXS \emph{member institutions} are all institutions
with an SXS member.

New members must send an e-mail to the executive committee
(\emph{exec@black-holes.org}) acknowledging that they have read and
agree to these SXS policies, including the \emph{code of
  conduct}~(\S\ref{sec:code_of_conduct}) and \emph{publication
  policies}~(\S\ref{sec:publication_policies}).  For example:

\begin{displayquote}
  Dear SXS executive committee,

  I, \textbf{NAME}, acknowledge that I have read and agree with the
  SXS policies as outlined in the SXS Collaboration Policies
  document.

  Best,

  \textbf{NAME}
\end{displayquote}

Faculty or senior researchers who are not members of the SXS
collaboration can be invited by the SXS executive committee to join
the SXS Collaboration.  Prior to extending the invitation, an
announcement must be sent to \emph{sxs-announce@black-holes.org} and
the \texttt{\#general} channel of the SXS collaboration Slack
(sxscollab.slack.com), giving members a week to share any concerns
about the new potential member.  Concerns may be sent either to the
executive committee or anonymously to the
\emph{ombudsperson}~(\S\ref{sec:ombudsperson}) or
\emph{advocate}~(\S\ref{sec:advocate}).  Suggested text:

\begin{displayquote}
  Dear SXS members,

  The SXS executive committee is looking for feedback on extending an
  invitation to \textbf{NAME} to join SXS as a faculty/senior
  researcher member.  Please provide feedback either directly to the
  executive committee at exec@black-holes.org, or anonymously through
  the SXS ombudsperson/advocate at ombudsperson@black-holes.org
  or sxs-advocate@black-holes.org by \textbf{DATE}.

  Best,

  The SXS executive committee
\end{displayquote}

SXS student and postdoc members who move on to a non-SXS institution
as a postdoc must decide whether they want to remain as \emph{alumni
  members}, and retain access to some SXS collaboration resources and
adhere to SXS policies, or leave SXS and only retain access to public
SXS resources.  Regardless of the decision, any projects initiated
before leaving an SXS institution continue to be governed by SXS
policies until their completion, with continued access to any SXS
resources that are required for the project.  The student or postdoc
should discuss with their advisor whether or not they wish to remain
in the collaboration.  If the advisor and student/postdoc agree that
remaining in SXS will be mutually beneficial, the advisor will ask the
executive committee for approval.  The advisor will continue to
supervise, be responsible for, and remain the point of contact for the
student/postdoc.  Specifically, the advisor will be responsible for
ensuring correct access and usage of SXS resources such as computer
time, codes, and SXS private data.

If a member decides to leave the collaboration, they may retain and
use current copies of SXS codes, but are expected to offer authorship
to the developers of the codes who have earned authorship rights as
discussed in \S\ref{sec:publication_policies}.  The member leaving the
collaboration can also transition to being an external collaborator
and discuss with the executive committee what other SXS resources they
can use.  In all cases, members who leave may continue to use SXS
resources to finish any projects that were started as SXS members.
Additionally, any SXS code developers (see \S\ref{sec:codes}) who
leave the collaboration and are no longer contributing to the code
will retain any authorship rights they have for at least two years and
at least five papers using the code.

If a current SXS student, postdoc, or alumni member is starting a
faculty position, they may apply to the executive committee to become
a new senior member.  If the executive committee approves the request,
the executive committee with then follow the procedures above by
sending a collaboration-wide announcement.

SXS members can be members of other collaborations such as the LIGO
Scientific Collaboration, the LISA Consortium, etc.  If you believe
there is a conflict between SXS policies and the policies of the other
collaboration, please bring it to the attention of the executive
committee so that the conflict can be resolved.

\section{SXS executive committee}\label{sec:executive_committee}

The responsibility of oversight of the SXS collaboration rests with
the SXS \emph{executive committee}.

The SXS Executive Committee has the following authority and
  responsibilities:
\begin{enumerate}
\item It has the authority and responsibility for allocating the grant
  funds from our collaborative grants, with the allocations being
  guided by our scientific goals and by the various intermediate
  benchmarks that the Executive Committee defines.  For example, as
  spelled out in our current Fairchild funding proposal, the Executive
  Committee receives this authority from Saul Teukolsky, the Fairchild
  grant PI.  Saul, in consultation with the co-PI Kip Thorne, has the
  authority to overrule the Executive Committtee's funding allocation
  decisions for the Fairchild grant, though this will happen very
  rarely if ever.  The Fairchild funds support people and activities
  at Caltech and Cornell.  There are also funds for computing
  hardware. These hardware resources are open to all members of SXS.
  Note that grants for individual institutions are not overseen by the
  executive committee, but members should feel free to coordinate
  activities funded by these grants.
\item It has the authority and responsibility to keep itself informed
  about all research projects that are supported by our funds (their
  nature, progress, and anticipated duration).
\item It has the authority and responsibility to oversee and allocate
  SXS collaboration resources~(\S\ref{sec:resources}).
\item It has the authority and responsibility to oversee and interpret
  the policies outlined in this document, including the code of
  conduct~(\S\ref{sec:code_of_conduct}) and publication
  policies~(\S\ref{sec:publication_policies}).
\item It has the authority and responsibility to modify these policies
  and to devise new policies for anything that is not covered by this
  document.  SXS members may propose modifications or new policies by
  contacting a member of the executive committee or the advocate.  All
  changes to the policies in this document will be announced to the
  collaboration in order to obtain feedback prior to their adoption.
\item It has the authority to remove a member who is no longer using
  SXS resources or who has seriously violated the SXS Code of Conduct.
\end{enumerate}

The current SXS executive committee must decide when inviting a new
faculty or senior researcher to join the collaboration whether to
also invite them to join the executive committee.

In general, the SXS executive committee reaches decisions via majority
vote.  However, any decision that involves resources from a specific
grant can be vetoed by the PI or co-PI of that grant.

The SXS collaboration will continue to exist until it is unanimously
decided to disband the collaboration.

The Student/Postdoc advocate (\S\ref{sec:advocate}) is a liason to the
executive committee, and not a voting member.  The SXS executive
committee cannot assign the Advocate any tasks or
responsibilities. This is to ensure that the Advocate is representing
the students and postdocs, and is \textit{not} a member of the
executive committee.

\section{SXS code of conduct}\label{sec:code_of_conduct}

It is very important that SXS is a positive, welcoming, and supportive
environment for everyone.  Therefore, we have adopted a code of
conduct.  All SXS members must provide written (email okay) agreement
that they have read and will abide by the code of conduct (see example
text in \S\ref{sec:sxs_membership}).

SXS has created the positions of \emph{ombudsperson} and
\emph{student-postdoc advocate} with whom members can informally and
confidentially discuss any concerns that they might have regarding
conflicts, problems, violations of the code of conduct, or any other
concerns they have.  The current ombudsperson and student-postdoc
advocate are listed in Appendix~\ref{sec:executive_committee_members},
and can be reached at \emph{ombudsperson@black-holes.org} and
\emph{sxs-advocate@black-holes.org}

\subsection{Code of conduct text}

\subsubsection{Overview} 
With the goal of supporting our fellow group members in doing the best
science we can, we expect that all members of the Simulating eXtreme
Spacetimes collaboration
\begin{itemize} 
\item Behave professionally in a way that is welcoming and respectful
  to all participants.
\item Behave in a way that is free from any form of discrimination,
  harassment, or retaliation.
\item Treat each other with collegiality and respect and help to
  create a supportive working environment.
\end{itemize}
For more specific guidelines for appropriate conduct, we refer
collaboration members to the “Ground Rules” section below, which is
based on the LLVM Code of Conduct at
\url{https://llvm.org/docs/CodeOfConduct.html}.

If you observe violations of the code of conduct, or have concerns about
potential violations, we encourage you to formally report this to one or more
members of the SXS Executive Committee. After investigation, the Executive
Committee will work to resolve the situation, in consultation with parties
involved and preserving confidentiality to the extent allowed by law, as
outlined below. 

If you have a concern about a code of conduct violation or potential
violation, but you do not wish to report it formally, you can speak
informally and confidentially about your concern with the SXS
Ombudsperson (see below).

The Executive Committee will take appropriate action to ensure that
all members of the collaboration meet the expectations of our code of
conduct.

\subsubsection{Ombudsperson}
\label{sec:ombudsperson}
The Ombudsperson (currently Geoffrey Lovelace, email
\emph{ombudsperson@black-holes.org}) is a senior member of the collaboration, 
either faculty or a senior researcher, who offers confidential, neutral, and
informal conflict resolution services to members of the SXS
Collaboration. All collaboration members should feel free to contact
the Ombudsperson to informally and confidentially discuss any concerns
that they might have regarding conflicts, problems, violations of the
code of conduct, or any other concerns they might have.

After speaking with Ombudsperson, if you wish, you can choose to ask
the Ombudsperson to bring your concerns to the attention of the SXS
Executive Committee or to other people at the appropriate
institutions, and you can also choose to report your concerns formally
to the SXS Executive Committee (as outlined below). Otherwise, unless
there is a serious issue of safety, your conversations with the
Ombudsperson will remain confidential.

We have adopted the same policy as the LIGO Scientific Collaboration
Ombudsperson; see this document, replacing “LSC” with “SXS
Collaboration”, and “LSC Spokesperson” with “the SXS Executive
Committee”, for more details about the Ombudsperson’s role:
\url{https://dcc.ligo.org/public/0099/M1300006/001/LSCOmbudsperson.pdf}

The executive committee will solicit volunteers for the
position.  The executive committee will choose the ombudsperson by
holding an election with ranked-choice voting among those nominees
that agreed to stand for election.  The ombudsperson serves a
two-year term.

\subsubsection{Student/Postdoc Advocate}
\label{sec:advocate}
The Student/Postdoc Advocate (currently Masha Okounkova, email
\emph{sxs-advocate@black-holes.org}) is someone who will advocate
positively for the people in the collaboration with less power,
specifically high school students, undergraduate students, summer REU
students, grad students, and postdocs. It may sometimes be daunting to
go to professors, so please feel free to talk to the Advocate about
any issues and concerns you may have in the collaboration, especially
concerns about culture and environment. The Advocate can bring up
(anonymously) any of your concerns with the SXS executive committee if
you wish. Otherwise, unless there’s a serious issue of safety, the
Advocate will keep the Ombudsperson in the loop but otherwise keep
your conversation confidential.


The SXS Advocate must be elected by the student and postdoc members of
SXS, and be within two years of their Ph.D.~(i.e.~a senior graduate
student or a new postdoc).  The SXS executive committee will have a
call for nominations and then consult with nominees to see if they are
willing to be the advocate.  If the SXS executive committee has any
concerns with a nominee, they will discuss these concerns with the
nominator.  The SXS executive committee will then supervise an
election among SXS student and postdoc members with ranked-choice
voting among those nominees that agreed to stand for election.  Once
the list of candidates for SXS advocate is announced, the executive
committee cannot change the candidate list (unless there is a serious
violation of the SXS code of conduct).  The Advocate is elected for a
one-year term, and may be re-elected.

\subsubsection{Ground Rules (following the LLVM Code of Conduct)} 
The Simulating eXtreme Spacetimes (SXS) Collaboration has the goal of
supporting our fellow group members in doing the best science we
can. To this end, we have a few ground rules that we ask people to
adhere to:
\begin{itemize}
  \item be friendly and patient, 
  \item be welcoming, 
  \item be considerate, 
  \item be respectful, 
  \item be careful in the words that you choose and be kind to others, and
  \item when we disagree, try to understand why. 
\end{itemize}

This is not an exhaustive list of things that you can't do. Rather,
take it in the spirit in which it is intended — a guide to make it
easier to communicate and participate in the community. This code of
conduct applies to all spaces managed by the SXS Collaboration. This
includes physical spaces at institutions with SXS group members; Slack
channels, mailing lists, and bug trackers; SXS Collaboration events
(such as the visiting institutions and workshops); and any other
forums created by the collaboration uses for communication. It applies
to all of your communication and conduct in these spaces, including
emails, chats, things you say, slides, videos, posters, signs, or even
t-shirts you display in these spaces. In addition, violations of this
code outside these spaces may, in rare cases, affect a person’s
ability to participate within them, when the conduct amounts to an
egregious violation of this code. If you believe someone is violating
the code of conduct, we ask that you report it by emailing one or more
members of the SXS Executive Committee (see “Reporting” below). If you
would rather not formally report your concern, you should feel free to
discuss it informally and confidentially with the SXS Ombudsperson.

\begin{itemize}
\item \textbf{Be friendly and patient.} During teleconferences and
  meetings, participants who wish to speak should feel free to type
  “hand up” or similar in the comment box, as needed, to get the
  chairs’ attention. Meeting/teleconference chairs are encouraged to
  make space for those unfamiliar with the topic of discussion to ask
  questions and engage.
\item \textbf{Be welcoming.} We strive to be a community that welcomes
  and supports people of all backgrounds and identities. This
  includes, but is not limited to members of any race, ethnicity,
  culture, national origin, colour, immigration status, social and
  economic class, educational level, sex, sexual orientation, gender
  identity and expression, age, size, family status, political belief,
  religion or lack thereof, and mental and physical ability.
\item \textbf{Be considerate.} Your work will be used by other people,
  and you in turn will depend on the work of others. Any decision you
  take will affect users and colleagues, and you should take those
  consequences into account. Remember that we’re a world-wide
  community, so you might not be communicating in someone else’s
  primary language.
\item \textbf{Be respectful.} Not all of us will agree all the time,
  but disagreement is no excuse for poor behavior and poor manners. We
  might all experience some frustration now and then, but we cannot
  allow that frustration to turn into a personal attack. It’s
  important to remember that a community where people feel
  uncomfortable or threatened is not a productive one. Members of the
  SXS Collaboration should be respectful when dealing with other
  members as well as with people outside the SXS Collaboration.
\item \textbf{Be careful in the words that you choose and be kind to
    others.} Do not insult or put down other participants. Harassment
  and other exclusionary behavior aren’t acceptable. This includes,
  but is not limited to:
  \begin{itemize}
  \item Violent threats or language directed against another person.
  \item Discriminatory jokes and language.
  \item Posting sexually explicit or violent material.
  \item Posting (or threatening to post) other people’s personally
    identifying information (“doxing”).
  \item Personal insults, especially those using racist or sexist
    terms.
  \item Unwelcome sexual attention.
  \item Advocating for, or encouraging, any of the above behavior.
  \end{itemize}
\item \textbf{In general, if someone asks you to stop, then stop.}
  Persisting in such behavior after being asked to stop is considered
  harassment.
\item \textbf{When we disagree, try to understand why.} Disagreements,
  both social and technical, happen all the time, and SXS is no
  exception. It is important that we resolve disagreements and
  differing views constructively.  Remember that we’re different. The
  strength of communities comes from having a varied community, people
  from a wide range of backgrounds. Different people have different
  perspectives on issues. Being unable to understand why someone holds
  a viewpoint doesn’t mean that they’re wrong. Don’t forget that it is
  human to err and blaming each other doesn’t get us
  anywhere. Instead, focus on helping to resolve issues and learning
  from mistakes.
\end{itemize}

\subsubsection{Reporting (following the LLVM reporting process)} 

If you believe someone is violating the code of conduct, you can
always file a report by emailing one or more members of the SXS
Executive Committee. \textbf{All reports will be kept confidential.}

If you believe anyone is in \textbf{physical danger}, please notify
appropriate law enforcement first. If you are unsure what law
enforcement agency is appropriate, please include this in your report
and we will attempt to notify them.

Reports of violations of the code of conduct can be as formal or
informal as needed for the situation at hand. If possible, please
include as much information as you can. If you feel comfortable,
please consider including:
\begin{itemize} 
\item Your contact info (so we can get in touch with you if we need to
  follow up).
\item Names (real, nicknames, or pseudonyms) of any individuals
  involved. If there were other witnesses besides you, please try to
  include them as well.
\item When and where the incident occurred. Please be as specific as
  possible.
\item Your account of what occurred. If there is a publicly available
  record (e.g. a mailing list archive or Slack logs) please include a
  link.
\item Any extra context you believe existed for the incident.
\item If you believe this incident is ongoing.
\item Any other information you believe we should have.
\end{itemize}

What happens after you file a report? (Following the LLVM process) —
You will receive an email from the SXS Executive Committee member(s)
you contacted, acknowledging receipt within 24 business hours (and we
will aim to respond much quicker than that).  If you do not receive an
acknowledgment, please resend your report.

They will review the incident and try to determine: 
\begin{itemize}
\item What happened and who was involved.
\item Whether this event constitutes a code of conduct violation.
\item Whether this is an ongoing situation, or if there is a threat to
  anyone’s physical safety.
\end{itemize}

Once the contacted SXS Executive Committee members
have a complete account of the events they will make a recommendation to the
full Executive Committee as to how to respond. Responses may include: 
\begin{itemize}
\item Nothing, if we determine no violation occurred or it has already
  been appropriately resolved.
\item Providing either moderation or mediation to ongoing interactions
  (where appropriate, safe, and desired by both parties).
\item A private reprimand from the working group to the individuals
  involved.
\item An imposed vacation (i.e.  asking someone to take a week off
  from a mailing list or Slack).
\item Escalation to the appropriate institutions.
\item Involvement of relevant law enforcement if appropriate.
\end{itemize}
If the situation is not resolved within one week, we will respond
within one week to the original reporter with an update and
explanation. Once we have determined our response, we will separately
contact the original reporter and other individuals to let them know
what actions (if any) will be taken. We will take into account
feedback from the individuals involved on the appropriateness of our
response, but we don’t guarantee we’ll act on it.




\section{SXS scientific goals and data}\label{sec:goals_and_data}

The scientific goals of the SXS collaboration are to develop codes
that can be used to predict the gravitational waveforms and
multi-messenger signals (electromagnetic and neutrino) from
astrophysical sources targeted by current and future gravitational
wave detectors.  These sources include compact binary mergers,
core-collapse supernovae, and accretion disks about compact objects
(e.g.~neutron stars and black holes).

The codes developed by members of the collaboration in pursuit of the
aforementioned goals (whether they perform the numerical simulations
of astrophysical sources, extract predictions of waveforms or signals
from such sources, or are used to build models of the signals from
such sources) are considered collaboration codes, and as such are
available for all members of the collaboration.  This applies whether
or not the codes are open source.

Contributing to or using non-SXS codes does not fall under these
policies.  However, the executive committee must be consulted before
using SXS computational resources with non-SXS codes.

The data produced by SXS codes in pursuit of the above scientific
goals are considered collaboration data and are available for all
members of the collaboration to use.  Periodically the collaboration
will release public catalogs of collaboration data, after which the
data will be considered public data.

If SXS resources~(\S\ref{sec:resources}) are used for other goals
(e.g.~proto-planetary disks, or white dwarf mergers, or LIGO thermal
noise calculations), then the following apply:
\begin{itemize}
\item The SXS executivex committee can decide whether or not they
  wish to supervise the project.
\item The publication and data policies of the indvidual SXS codes
  will still apply to the project.
\item It is expected that code improvements for such purposes be
  merged into the SXS code repositories.
\item Data produced by such projects are not considered collaboration
  data, unless the producers decide to make it collaboration data and
  the executive committee is willing to maintain and/or host the data.
\end{itemize}

\section{SXS projects}\label{sec:projects}

Any scientific research (with the expectation of producing one or more
publications) that uses SXS resources~(\S\ref{sec:resources}) is
designated an \emph{SXS project} subject to oversight by the SXS
executive committee.  Any project that uses only public data from the
start of the project will not be designated an SXS project.  When
starting a new project, SXS members must consult with a member of the
SXS executive committee (who may consult the entire executive
committee and/or lead code developers) in order to determine whether
or not their project is considered an SXS project.  The executive
committee may relinquish oversight of a project if the decision to do
so is unanimous amongst the executive committee, in which case the
project is not considered to be an SXS project.

At the initiation of an SXS project, an electronic announcement must
be sent to the SXS Announce mailing list
\emph{sxs-announce@black-holes.org}, announcing the project,
summarizing the project and its scope in brief, listing those who are
initially involved, and inviting other SXS members to join.  The SXS
Announce mailing list consists of all SXS members.  Early and timely
announcement of projects is vital to the health of the collaboration
and to maintaining a collegial environment. Those leading the effort
on the project are expected to provide periodic updates to an
executive committee member and are strongly encouraged to present
updates at one or more of the weekly telecons.  Each project should
outline the expectations of each contributor in order to earn
authorship rights on publications from the project.

Although early project announcement is important, it is not intended
as a method of ``fencing off'' scientific territory. SXS members are
encouraged to collaborate and communicate with other interested
members on analysis projects, and the participation of those wishing
to join an analysis project should be welcomed assuming there is
sufficient work to be done. At the same time, there may be cases in
which multiple, independent analysis work on the same or similar
topics is appropriate or even desirable. In these cases, the executive
committee will be responsible for ensuring sufficient coordination of
the analyses and resulting publications.


\section{SXS resources}\label{sec:resources}

In order to achieve its scientific goals, the SXS collaboration
provides a set of community resources for all SXS members.  These
include a set of \emph{SXS codes} whose \emph{developers} nay earn
authorship rights on publications from a subset of SXS projects, and
\emph{SXS infrastructure} whose \emph{maintainers} may also earn
authorship rights as described below.

\subsection{Codes and developers}\label{sec:codes}

Codes developed to produce collaboration data must be made available
to all SXS members.  In order to recognize the significant work of
code developers, the executive committee may designate a code as a SXS
collaboration code which gives some of its developers authorship
rights on a subset of SXS projects that either use the code or
non-public data generated by the code. The lead developers of a code
may submit a request to the executive committee that their code be
designated as an SXS collaboration code.  The executive committee may
also promote codes developed within the collaboration to be a
collaboration code so that there is someone to take them over in case
the primary developer leaves academia.  The executive committee,
however, can not designate external codes to be SXS codes.  In
addition, the executive committee can demote an existing SXS code if
it is no longer maintained.

Once designated as a collaboration code by the executive committee,
the lead developers of the code must present and receive executive
committee approval of a management plan that:
\begin{itemize}
\item Explains how the code will be made accessible to all SXS
  members.  For example, as either a public repository or a private
  repository in the GitHub sxs-collaboration organization.
\item Describes how the code is documented, and how SXS members can
  ask for help in using the code.
\item Describes how others can contribute to the code.
\item Designates one to three lead developers who will be the primary
  contacts for interaction with the executive committee, and for
  inquiries from SXS members.  Note that a member of the executive
  committee is permitted be a lead developer.
\item Describes the publication policies for papers that use the code
  or non-public data produced by the code.  We provide a default
  publication policy in~\S\ref{sec:publication_policies}.
\end{itemize}
Otherwise, developers are free to organize their project as they see
fit.  Developers are free to modify their publication policies,
subject to approval from the executive committee.

Developers are expected to provide reasonable support for existing
code, including improving documentation and fixing bugs on a
reasonable timescale.  Developers, however, are not required to
implement new features, nor spend unreasonable time supporting new
code development by others.

Note that simple codes and scripts will not be designated as
collaboration codes.  Only codes with significant development and
maintenance will become collaboration codes.  Please discuss with a
member of the executive committee any development plans for any codes
beyond simple plotting scripts and data analysis tools.

The advantages of being designated a code as a collaboration code are
several:
\begin{itemize}
\item Some developers of a collaboration code are given publication
  rights on projects that use the code or data produced by the code.
\item Additional SXS members can contribute to the code leading to
  significant improvements for all SXS members.
\item The collaboration will be able to ensure the existence of the
  code beyond the interests of the original developer.
\end{itemize}

Most SXS members will contribute to SXS codes in a modest fashion
during work towards their projects.  However, a small number of SXS
members put in significant effort into building the infrastructure of
a SXS code. In order to reward their effort, an SXS member can be
designated to have earned authorship rights on papers using the SXS
code.  Each collaboration code should outline the requirements for
someone being designated as having earned authorship rights.  For
large codes such as SpEC or SpECTRE, the suggestion is a total of
approximately 3000 hours of effort.  For large codes, authorship
rights may be granted for a subset of the code.


In addition, an SXS member can be earn authorship rights for making a
significant contribution to the code, as decided by the existing
developers of the code.  If the current developers do not deem the
contribution significant enough to earn authorship rights, the SXS
member may appeal to the executive committee.  The executive committee
will discuss the appeal with the current developers and, if they so
choose, deem the contribution significant. 
The following all count as code infrastructure work:
\begin{itemize}
\item Maintenance, refactoring, and modernization of source code,
  build systems, continuous integration, and source repositories
\item Fixing bugs and optimizing code
\item Performing code review
\item Adding or maintaining the ability to build the SXS code on
  high performance computing systems that SXS uses
\item Development and implementation of code that is
  merged into the SXS accessible repository, clearly documented,
  well-tested to ensure correctness over time, and align with the
  scientific goals of the collaboration.
\item Development of backend code that allows the algorithms to work
  together to perform simulations (e.g.~parallelization code like MPI,
  I/O code, widely used data structures, etc.)
\item Running workshops about the code or concepts used by the code.
\end{itemize}

The list of developers with authorship rights for each code shall be
maintained by the executive committee.  Nominations for new authorship
rights based on contributing development work shall be brought forward
to the executive committee at the beginning of January, May, and
September of each year. Approval of authorship rights shall be carried
out by the executive committee with consultation of existing
developers. Once given authorship rights, that status is maintained as
long as the person remains a member or participant in the SXS
collaboration.

The current list of SXS codes and their lead developers are listed in
Appendix~\ref{sec:current_codes}.

\subsection{Infrastructure and maintainers}\label{sec:infrastructure}
The SXS collaboration provides a set of community resources for all
members.  These include:
\begin{itemize}
\item SXS website \url{https://www.black-holes.org}
\item SXS waveform catalog
\item SXS GitHub organization \url{https://github.com/sxs-collaboration}
\item SXS Slack organization
\item SXS compute clusters
\item SXS continuous integration computers
\item SXS Google organization and mailing lists
\item SXS allocation on national computing clusters
\item Organization of workshops about SXS research
\end{itemize}

SXS members who spend time as maintainers of these resources can count
the time towards earning authorship rights for the code of their
choice, unless it is clearly contributing to a specific code.  SXS
members are encouraged to contribute to the maintenance of
infrastructure, and should contact their local member of the executive
committee in order to explore opportunities for how to contribute.

A complete list of SXS resources and their maintainers are listed in
Appendix~\ref{sec:current_infrastructure}.  If you believe there is an
SXS resource missing from the list of resources, please contact a
member of the executive committee in order to discuss if it should be
included.

\subsection{Public outreach\label{sec:public outreach}}

SXS views public outreach and engagement as an essential activity for
SXS members to participate in. To this end, public outreach is counted
as infrastructure work and is counted towards the time needed for
earning authorship rights. The time can be allotted to any SXS
project/code desired. Outreach activities include but are not limited
to:
\begin{itemize}
\item helping with science olympiads,
\item generating animations and videos for the SXS YouTube channel, or LIGO/LVK
  to use,
\item being a panelist at outreach events,
\item advocacy work related but not limited to mental health, diversity, equity,
  inclusion, accessibility, etc.~in STEM/academia,
\item attending schools to teach children about science, even if it is not
  related to the SXS Scientific Goals,
\item running and helping with workshops for undergraduate or graduate students,
  for example as part of a Research Experience for Undergraduates program,
\item volunteering at science centers, science museums, or planetariums.
\end{itemize}

\section{SXS publication policies}\label{sec:publication_policies}

The SXS Publication Policy is designed to promote the scientific and
technical accuracy and timeliness of SXS publications and to ensure
that fair credit is given to the authors and to other individuals who
have contributed to the resources used by a SXS project.

\subsection{Purview of this Policy}

This policy covers all SXS projects unless modifications have been
approved by the executive committee during the SXS project.  The
developers of each SXS code may choose to adopt these policies with
respect to authorship rights for SXS projects that use the code or
data produced by the code, including modifications approved by the
executive committee.  The policies apply to use of SXS codes,
regardless of the copyright and license of the code, or whether or not
the code is publicly available. It also applies to any data generated
using a SXS code that were not public at the time the SXS Project
began, even if those data become public before completion of the
analysis and/or the resulting publication.

Disputes about publication matters, including but not limited to
authorship and author ordering, will be referred to the executive
committee if they cannot be resolved directly with the lead authors.

\subsection{Types of papers}

We distinguish several types of papers, which are governed by different
guidelines, as discussed in later Sections of this policy:
\begin{enumerate}
\item Scientific publications presenting results using one or more SXS
  codes. These publications are further subdivided into:
  \begin{itemize}
  \item Collaboration science papers that involve results highlighting
    major accomplishments for the core science goals of the SXS
    collaboration. For example, catalog papers, announcements of new
    waveforms, or highlights of major breakthroughs that SXS as a
    whole has achieved.
  \item Standard science papers. Examples include surrogates built
    using private waveforms, efficacy of new types of initial data,
    efficacy of new gauge conditions, critical collapse, and
    simulations in beyond-GR theories.
  \end{itemize}
  
\item Technical papers describing specific numerical
  algorithms. Technical papers do not show new scientific results, but
  may show some representative simulations compared to the existing
  methods. An example would be a paper describing local time stepping,
  demonstrating speedup and parallel scaling on various problems
  compared to the existing code.
\end{enumerate}

It should be made clear what the scope of the project is from the
beginning in order to decide the type of paper.  If the scope
significantly changes during the project, the executive committee must
be updated and may deem the project to fall under a different paper
category.

\subsection{Authorship and author list ordering}

Authorship of SXS science publications using SXS codes will be drawn
from two groups: (i) those members and external collaborators who
contributed to the analysis and writing of the paper (referred to
below as primary authors); and (ii) those designated developers for
the SXS codes and maintainers of SXS resources used by the project.

For Collaboration science papers, the author list shall be The SXS
Collaboration followed by the author names in alphabetical order. In
general, anyone who is a member of SXS and has contributed to the
results being presented has rights to authorship.  This includes all
designated developers of any SXS codes used to produce the results, as
well as any designated maintainers of SXS infrastructure.
Furthermore, someone who has run even a single simulation that is part
of a catalog paper will be an author.

For standard science papers, the default ordering will include two
tiers, the primary authors and analyzers, followed by an alphabetical
listing of those SXS code developers with authorship rights who have
requested authorship. The author ordering within the first tier is at
the discretion of the lead authors of the paper. If they wish, the
lead authors can opt for alphabetical ordering within the first tier
or for alphabetical ordering of the entire list.  An invitation to SXS
code developers with authorship rights must be extended at least one
week prior to writing of the initial draft in order to have the
opportunity to meaningfully contribute to the paper.  In addition, SXS
code developers with authorship rights must be notified at least one
week before submission of the paper so they have sufficient time to
decide upon whether or not they wish to be an author on the paper.
Please note the collaborative process is enhanced the earlier
invitations are made.

For technical papers, authorship will generally be confined to those
who made direct contributions to the paper, the lead authors shall
decide on the author ordering. Again, they can choose to make the
ordering alphabetical if they wish.  In addition, the authors of such
technical papers must be given authorship rights to at least the first
three scientific applications of the methods or results of the
technical paper.

The provisions of this authorship section may be superseded in cases of
violation of the Code of Conduct.

\appendix

\section{SXS membership}
\subsection{Faculty and senior researchers}\label{sec:member_faculty}

\begin{itemize}
\item Nils Deppe (Cornell University)
\item Matthew Duez (Washington State University)
\item Scott Field (University of Massachusetts Dartmouth)
\item Francois Foucart (University of New Hampshire)
\item Lawrence Kidder (Cornell University)
\item Geoffrey Lovelace (California State University, Fullerton)
\item Elias Most (California Institute of Technology)
\item Harald Pfeiffer (Max Planck Institute for Gravitational Physics)
\item Mark Scheel (California Institute of Technology)
\item Leo Stein (University of Mississippi)
\item Saul Teukolsky (California Institute of Technology and Cornell University)
\item Vijay Varma (University of Massachusetts Dartmouth)
\item Aaron Zimmerman (University of Texas)
\end{itemize}

\subsection{Executive committee}\label{sec:executive_committee_members}
\begin{itemize}
\item Executive committee (\emph{exec@black-holes.org}): Deppe, Duez,
  Foucart, Kidder, Lovelace, Most, Pfeiffer, Scheel, Stein, Teukolsky,
  Varma, Zimmerman
\item Ombudsperson (\emph{ombudsperson@black-holes.org}): Geoffrey Lovelace
\item Student-Postdoc Advocate (\emph{sxs-advocate@black-holes.org}):
  Marceline (Marcie) Bonilla
\end{itemize}

\section{SXS codes and developers}\label{sec:current_codes}
\subsection{Code name}
\begin{itemize}
\item How to obtain
\item Lead developer contact info
\item Developers
\item Desired acknowledgment
\end{itemize}
\subsection{SpEC}
\subsection{SpECTRE}


\section{SXS infrastructure and
  maintainers}\label{sec:current_infrastructure}
\subsection{SXS resource}
\begin{itemize}
\item where to find it
\item who to contact
\item standard acknowledgment (if applicable)
\end{itemize}
\subsection{ SXS website}
\subsection{ SXS waveform catalog}
\subsection{ SXS GitHub organization}
This can be queried from GitHub:\\
https://github.com/orgs/sxs-collaboration/people?query=role\%3Aowner
\subsection{ SXS Slack organization}
\subsection{ SXS compute clusters}
\subsection{ SXS CI computers}
\subsection{ SXS Google organization/mailing lists}
\subsection{SXS computer allocations}


\end{document}



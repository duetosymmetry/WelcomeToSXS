\subsubsection{Overview} 
With the goal of supporting our fellow group members in doing the best
science we can, we expect that all members of the Simulating eXtreme
Spacetimes collaboration
\begin{itemize} 
\item Behave professionally in a way that is welcoming and respectful
  to all participants.
\item Behave in a way that is free from any form of discrimination,
  harassment, or retaliation.
\item Treat each other with collegiality and respect and help to
  create a supportive working environment.
\end{itemize}
For more specific guidelines for appropriate conduct, we refer
collaboration members to the “Ground Rules” section below, which is
based on the LLVM Code of Conduct at
\url{https://llvm.org/docs/CodeOfConduct.html}.

If you observe violations of the code of conduct, or have concerns about
potential violations, we encourage you to formally report this to one or more
members of the SXS Executive Committee. After investigation, the Executive
Committee will work to resolve the situation, in consultation with parties
involved and preserving confidentiality to the extent allowed by law, as
outlined below. 

If you have a concern about a code of conduct violation or potential
violation, but you do not wish to report it formally, you can speak
informally and confidentially about your concern with the SXS
Ombudsperson (see below).

The Executive Committee will take appropriate action to ensure that
all members of the collaboration meet the expectations of our code of
conduct.

\subsubsection{Ombudsperson}
\label{sec:ombudsperson}
The Ombudsperson (currently Geoffrey Lovelace, email
\emph{ombudsperson@black-holes.org}) is a senior member of the collaboration, 
either faculty or a senior researcher, who offers confidential, neutral, and
informal conflict resolution services to members of the SXS
Collaboration. All collaboration members should feel free to contact
the Ombudsperson to informally and confidentially discuss any concerns
that they might have regarding conflicts, problems, violations of the
code of conduct, or any other concerns they might have.

After speaking with Ombudsperson, if you wish, you can choose to ask
the Ombudsperson to bring your concerns to the attention of the SXS
Executive Committee or to other people at the appropriate
institutions, and you can also choose to report your concerns formally
to the SXS Executive Committee (as outlined below). Otherwise, unless
there is a serious issue of safety, your conversations with the
Ombudsperson will remain confidential.

We have adopted the same policy as the LIGO Scientific Collaboration
Ombudsperson; see this document, replacing “LSC” with “SXS
Collaboration”, and “LSC Spokesperson” with “the SXS Executive
Committee”, for more details about the Ombudsperson’s role:
\url{https://dcc.ligo.org/public/0099/M1300006/001/LSCOmbudsperson.pdf}

The executive committee will solicit volunteers for the
position.  The executive committee will choose the ombudsperson by
holding an election with ranked-choice voting among those nominees
that agreed to stand for election.  The ombudsperson serves a
two-year term.

\subsubsection{Student/Postdoc Advocate}
\label{sec:advocate}
The Student/Postdoc Advocate (currently Masha Okounkova, email
\emph{sxs-advocate@black-holes.org}) is someone who will advocate
positively for the people in the collaboration with less power,
specifically high school students, undergraduate students, summer REU
students, grad students, and postdocs. It may sometimes be daunting to
go to professors, so please feel free to talk to the Advocate about
any issues and concerns you may have in the collaboration, especially
concerns about culture and environment. The Advocate can bring up
(anonymously) any of your concerns with the SXS executive committee if
you wish. Otherwise, unless there’s a serious issue of safety, the
Advocate will keep the Ombudsperson in the loop but otherwise keep
your conversation confidential.


The SXS Advocate must be elected by the student and postdoc members of
SXS, and be within two years of their Ph.D.~(i.e.~a senior graduate
student or a new postdoc).  The SXS executive committee will have a
call for nominations and then consult with nominees to see if they are
willing to be the advocate.  If the SXS executive committee has any
concerns with a nominee, they will discuss these concerns with the
nominator.  The SXS executive committee will then supervise an
election among SXS student and postdoc members with ranked-choice
voting among those nominees that agreed to stand for election.  Once
the list of candidates for SXS advocate is announced, the executive
committee cannot change the candidate list (unless there is a serious
violation of the SXS code of conduct).  The Advocate is elected for a
one-year term, and may be re-elected.

\subsubsection{Ground Rules (following the LLVM Code of Conduct)} 
The Simulating eXtreme Spacetimes (SXS) Collaboration has the goal of
supporting our fellow group members in doing the best science we
can. To this end, we have a few ground rules that we ask people to
adhere to:
\begin{itemize}
  \item be friendly and patient, 
  \item be welcoming, 
  \item be considerate, 
  \item be respectful, 
  \item be careful in the words that you choose and be kind to others, and
  \item when we disagree, try to understand why. 
\end{itemize}

This is not an exhaustive list of things that you can't do. Rather,
take it in the spirit in which it is intended — a guide to make it
easier to communicate and participate in the community. This code of
conduct applies to all spaces managed by the SXS Collaboration. This
includes physical spaces at institutions with SXS group members; Slack
channels, mailing lists, and bug trackers; SXS Collaboration events
(such as the visiting institutions and workshops); and any other
forums created by the collaboration uses for communication. It applies
to all of your communication and conduct in these spaces, including
emails, chats, things you say, slides, videos, posters, signs, or even
t-shirts you display in these spaces. In addition, violations of this
code outside these spaces may, in rare cases, affect a person’s
ability to participate within them, when the conduct amounts to an
egregious violation of this code. If you believe someone is violating
the code of conduct, we ask that you report it by emailing one or more
members of the SXS Executive Committee (see “Reporting” below). If you
would rather not formally report your concern, you should feel free to
discuss it informally and confidentially with the SXS Ombudsperson.

\begin{itemize}
\item \textbf{Be friendly and patient.} During teleconferences and
  meetings, participants who wish to speak should feel free to type
  “hand up” or similar in the comment box, as needed, to get the
  chairs’ attention. Meeting/teleconference chairs are encouraged to
  make space for those unfamiliar with the topic of discussion to ask
  questions and engage.
\item \textbf{Be welcoming.} We strive to be a community that welcomes
  and supports people of all backgrounds and identities. This
  includes, but is not limited to members of any race, ethnicity,
  culture, national origin, colour, immigration status, social and
  economic class, educational level, sex, sexual orientation, gender
  identity and expression, age, size, family status, political belief,
  religion or lack thereof, and mental and physical ability.
\item \textbf{Be considerate.} Your work will be used by other people,
  and you in turn will depend on the work of others. Any decision you
  take will affect users and colleagues, and you should take those
  consequences into account. Remember that we’re a world-wide
  community, so you might not be communicating in someone else’s
  primary language.
\item \textbf{Be respectful.} Not all of us will agree all the time,
  but disagreement is no excuse for poor behavior and poor manners. We
  might all experience some frustration now and then, but we cannot
  allow that frustration to turn into a personal attack. It’s
  important to remember that a community where people feel
  uncomfortable or threatened is not a productive one. Members of the
  SXS Collaboration should be respectful when dealing with other
  members as well as with people outside the SXS Collaboration.
\item \textbf{Be careful in the words that you choose and be kind to
    others.} Do not insult or put down other participants. Harassment
  and other exclusionary behavior aren’t acceptable. This includes,
  but is not limited to:
  \begin{itemize}
  \item Violent threats or language directed against another person.
  \item Discriminatory jokes and language.
  \item Posting sexually explicit or violent material.
  \item Posting (or threatening to post) other people’s personally
    identifying information (“doxing”).
  \item Personal insults, especially those using racist or sexist
    terms.
  \item Unwelcome sexual attention.
  \item Advocating for, or encouraging, any of the above behavior.
  \end{itemize}
\item \textbf{In general, if someone asks you to stop, then stop.}
  Persisting in such behavior after being asked to stop is considered
  harassment.
\item \textbf{When we disagree, try to understand why.} Disagreements,
  both social and technical, happen all the time, and SXS is no
  exception. It is important that we resolve disagreements and
  differing views constructively.  Remember that we’re different. The
  strength of communities comes from having a varied community, people
  from a wide range of backgrounds. Different people have different
  perspectives on issues. Being unable to understand why someone holds
  a viewpoint doesn’t mean that they’re wrong. Don’t forget that it is
  human to err and blaming each other doesn’t get us
  anywhere. Instead, focus on helping to resolve issues and learning
  from mistakes.
\end{itemize}

\subsubsection{Reporting (following the LLVM reporting process)} 

If you believe someone is violating the code of conduct, you can
always file a report by emailing one or more members of the SXS
Executive Committee. \textbf{All reports will be kept confidential.}

If you believe anyone is in \textbf{physical danger}, please notify
appropriate law enforcement first. If you are unsure what law
enforcement agency is appropriate, please include this in your report
and we will attempt to notify them.

Reports of violations of the code of conduct can be as formal or
informal as needed for the situation at hand. If possible, please
include as much information as you can. If you feel comfortable,
please consider including:
\begin{itemize} 
\item Your contact info (so we can get in touch with you if we need to
  follow up).
\item Names (real, nicknames, or pseudonyms) of any individuals
  involved. If there were other witnesses besides you, please try to
  include them as well.
\item When and where the incident occurred. Please be as specific as
  possible.
\item Your account of what occurred. If there is a publicly available
  record (e.g. a mailing list archive or Slack logs) please include a
  link.
\item Any extra context you believe existed for the incident.
\item If you believe this incident is ongoing.
\item Any other information you believe we should have.
\end{itemize}

What happens after you file a report? (Following the LLVM process) —
You will receive an email from the SXS Executive Committee member(s)
you contacted, acknowledging receipt within 24 business hours (and we
will aim to respond much quicker than that).  If you do not receive an
acknowledgment, please resend your report.

They will review the incident and try to determine: 
\begin{itemize}
\item What happened and who was involved.
\item Whether this event constitutes a code of conduct violation.
\item Whether this is an ongoing situation, or if there is a threat to
  anyone’s physical safety.
\end{itemize}

Once the contacted SXS Executive Committee members
have a complete account of the events they will make a recommendation to the
full Executive Committee as to how to respond. Responses may include: 
\begin{itemize}
\item Nothing, if we determine no violation occurred or it has already
  been appropriately resolved.
\item Providing either moderation or mediation to ongoing interactions
  (where appropriate, safe, and desired by both parties).
\item A private reprimand from the working group to the individuals
  involved.
\item An imposed vacation (i.e.  asking someone to take a week off
  from a mailing list or Slack).
\item Escalation to the appropriate institutions.
\item Involvement of relevant law enforcement if appropriate.
\end{itemize}
If the situation is not resolved within one week, we will respond
within one week to the original reporter with an update and
explanation. Once we have determined our response, we will separately
contact the original reporter and other individuals to let them know
what actions (if any) will be taken. We will take into account
feedback from the individuals involved on the appropriateness of our
response, but we don’t guarantee we’ll act on it.


